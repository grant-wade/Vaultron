\documentclass[12pt]{report}
% \usepackage[utf8]{inputenc} % pictures
\usepackage[titletoc]{appendix} % appendix
\usepackage[cm]{fullpage} % full page 
\usepackage{nopageno} % Disable bottom page #
\usepackage{multicol} % multicolumn 
\usepackage{enumitem} % list items
\usepackage{fancyhdr} % page headers
\usepackage{titlesec} % title format
\usepackage[margin=1in]{geometry}

%set font style
\renewcommand\familydefault{\sfdefault}

% Set line spacing
\renewcommand{\baselinestretch}{1}

% Set indent size
\setlength\parindent{0pt}

% Title format
\titleformat{\chapter}{\Huge\bf}{\thechapter}{20px}{\Huge\bf}


\begin{document}
% Begining of title page

{\Huge \textbf{Software Requirements Specification:}}
\vspace{5mm}
\begin{flushright}

    {\huge for}
    \vspace{20mm}

    \textbf{\Huge Vaultron}
    \vspace{20mm}

    {\huge Version $<0.0.1>$}
    \vspace{20mm}

    {\huge Prepared by}
    \vspace{20mm}

    \textbf{\huge Cryptomaniacs}
    \vspace{20mm}
\end{flushright}

\begin{multicols}{3}
    \begin{itemize}
        \item[] {\Large Colton King}
        \item[] {\Large Grant Wade}
        \item[] {\Large Robby Boney}
        \item[] {\Large Rob Wooner}
    \end{itemize}

    \begin{itemize}
        \item[] {\Large 11245746}
        \item[] {\Large 11435949}
        \item[] {\Large 11453444}
        \item[] {\Large 11496643}
    \end{itemize}

    \begin{itemize}
        \item[] {\Large\texttt colton.king@wsu.edu}
        \item[] {\Large\texttt grant.wade@wsu.edu}
        \item[] {\Large\texttt robby.boney@wsu.edu}
        \item[] {\Large\texttt robert.wooner@wsu.edu}
    \end{itemize}
\end{multicols}

\vfill

\begin{flushright}
    \vspace{20mm}
    {\Large \textbf{Date:} Sunday, October 15th, 2017}
\end{flushright}

% Begining of second page
\clearpage


\tableofcontents{}

\chapter{Introduction}
The goal of this project is to create a cryptographically
secure cross platform password manager. The cross platform
compatibility will be achieved using electron and node.js.
This section will describe who the intended audience for the
password manager will be and describe the purpose of the
project in depth.
\section{Document Purpose}
The product we are writing this SRS document for is the cryptographically
secure cross platform password manager Version 0.0.1 (Vaultron). This password manager 
will create strong passwords and encrypt them. It will remember the password 
for the website it is being created for.
\section{Project Scope}
This software is a password manager that creates cryptographically secure 
passwords. It will store the hashed passwords in a json file for safe keeping.
There can be multiple profiles, each one will have a master password of its own 
that will unlock the vault gaining access to the passwords. The master password
will be created by the user so that they can remember it. The password manager 
can however create a good master password that the user can write down to remember.
the master password will not be sent through email or text so that there will
be no chance of it being stolen from a malicious attacker.

Using this password manager allows the user to have strong and secure passwords 
that they will not have to remember. Not needing to remember allows for a strong 
password that has a very little chance of being cracked. The user only needs to sign
in with their master password and copy and paste the desired password from the 
vault into the website, or other password field.
\section{Intended Audience and Document Overview}
Client: The intended audience of this software is anyone that uses the internet 
and currently keeps track of their passwords by hand or lets the browser
keep track of them. This software will offer a safe, secure, and isolated
way to store any passwords. Security is our number one priority, if something
can be secured it will, so we can ensure that no user data can be leaked.

This document will be used as a starting point to overview the
software. It will describe how it can be used, the security behind the
password storage, and what technologies will be used to create the software.
The rest of the document contains all of the specifications for the
software, which includes product functionality, operating environment,
interface design, functional requirements, behavioral requirements, 
security requirements and other product details.
\section{Definitions, Acronyms and Abbreviations}
\textbf{CSS:} Cascading Style Sheets, used to style our electron application

\textbf{Electron:} Cross platform desktop application development environment
using HTML, CSS, and JavaScript

\textbf{Hardened Mode:} A mode within the product that enable encryption of entire
vault for transport

\textbf{HTML:} HyperText Markup Language, used to design our electron application

\textbf{JavaScript:} Programming language used on the web and Electron

\textbf{Vault:} Password storage platform created for this project

\section{Document Conventions}
This document follows the standard IEEE formatting.
We are doing nothing special in this SRS.

\section{References and Acknowledgments}

% END OF INTRODUCTION

\chapter{Overall Description}

\section{Product Perspective}
At the time of design this project is intended to fill a similar
niche in the password storage and serving field that \textit{1Password}
and \textit{LastPass} fill. Thus it is in the same family as
the two products previously mentioned, a system that stores passwords
securely and allows later retrieval. This product will be implemented in
two main parts. One, a desktop application that will serve as the \textit{vault}
storing the passwords, creating the passwords, and allowing later access
to the passwords. The second main part is the browser extensions that
will be used to auto-fill in usernames and passwords into any website
that the user adds to their vault. 

\section{Product Functionality}
The product is split into two components, desktop application and browser
extension. 

\subsection{Desktop Application}
\begin{itemize}
    \item Generate secure passwords (any length or complexity)
    \item Authenticate user to decrypt passwords (required before use)
    \item Generate a \textit{Vault} on first run of software
    \item Allow creation of new entries within the \textit{Vault} (passwords, notes, etc)
    \item Utility to sync users \textit{Vault} between computers (Dropbox, Google Drive, etc)
\end{itemize}
\subsection{Browser Extension}
\begin{itemize}
    \item Auto fill of information from the \textit{Vault} to the webpage
    \item Create of new entries in the \textit{Vault}
\end{itemize}

\section{Users and Characteristics}

\section{Operating Environment}
The environment in which this software will be operating in are all
major operating systems, OS, Windows, Linux. This is accomplished using
\textit{Electron} which allows us to create cross platform binaries
with the same codebase.

\section{Design and Implementation Constraints}
The biggest constraints for this software are time and security considerations.
We need to make sure that our software follows popular, effective, and
accepted security protocols. Given the time frame in which to create this
product and the importance that our password manager successfully encrypts
and protects our passwords, we need to work hard and efficient.

\section{User Documentation}
We will provide a simple help page that will go over how to use and run 
our software. The help page will go over how to create and add a new entry in the 
Vault. How to create a new password. How to delete an old entry. How to update 
an entry.

\section{Assumptions and Dependencies}
\begin{itemize}
    \item We assume that users will have at most 1000 passwords
    \item We assume that users will have a computer with minimum specifications
    \item We assume that the Node.js standard library won't drastically change
    \item We assume that Electron's standard library won't drastically change
    \item We assume that with a good master password, user's passwords will be secure
    
\end{itemize}



% END OF OVERALL DESCRIPTION

\chapter{Specific Requirements}

\section{External Interface Requirements}

\subsection{User Interfaces}
Our product will have a login window that will have a username text box and password
text box and an enter button. The enter button will be pressed after the password 
and username are inputted. The login window will blur out the rest of the window
behind it.
We will have a minimize, maximize  and exit button in the top left corner. 
we will have tabs along the top that will take you to different password profiles,
like work, play, etc. 
There will be a create password button that will create a pop up that has entries
for a url and a drop down for picking the password profile. 
\subsection{Hardware Interfaces}
We have no hardware interface that we need to run our software.

\subsection{Software Interfaces}

\subsection{Communications Interfaces}

\section{Functional Requirements}

\section{Behaviour Requirements}


% END OF SPECIFIC REQUIREMENTS


\chapter{Other Non-Functional Requirements}

\section{Performance Requirements}
The following requirements are based on a system with minimum specs:
\begin{itemize}
    \item Log-in time should not take more than 2 seconds (verifying user)
    \item New entries into the Vault should be less than 1 second before software can be used again
    \item Updating entries should take less than 1 second
    \item Password generation should take less than 1 second
\end{itemize}

\section{Safety and Security Requirements}


\section{Software Quality Attributes}


% END OF NONFUNCTIONAL REQUIREMENTS


\chapter{Other Requirements}

\begin{appendices}
    \chapter{Data Dictionary}
    \chapter{Group Log}
\end{appendices}

\end{document}

\documentclass{report}
% \usepackage[utf8]{inputenc} % pictures

\usepackage[titletoc]{appendix} % appendix
\usepackage[cm]{fullpage} % full page 
\usepackage{nopageno} % Disable bottom page #
\usepackage{multicol} % multicolumn 
\usepackage{enumitem} % list items
\usepackage{fancyhdr} % page headers

% Style choices, indent and text spacing
\renewcommand{\baselinestretch}{1}

% \rhead{Page \thepage}
% \lhead{Vaultron: Table of Contents}

\begin{document}
% Begining of title page

{\Huge \textbf{Software Requirements Specification:}}
\vspace{5mm}
\begin{flushright}

    {\huge for}
    \vspace{20mm}

    \textbf{\Huge Vaultron}
    \vspace{20mm}

    {\huge Version $<0.0.1>$}
    \vspace{20mm}

    {\huge Prepared by}
    \vspace{20mm}

    \textbf{\huge Cryptomaniacs}
    \vspace{20mm}
\end{flushright}

\begin{multicols}{3}
    \begin{itemize}
        \item[] {\Large Colton King}
        \item[] {\Large Grant Wade}
        \item[] {\Large Robby Boney}
        \item[] {\Large Rob Wooner}
    \end{itemize}

    \begin{itemize}
        \item[] {\Large 11245746}
        \item[] {\Large 11435949}
        \item[] {\Large 11453444}
        \item[] {\Large 11496643}
    \end{itemize}

    \begin{itemize}
        \item[] {\Large colton.king@wsu.edu}
        \item[] {\Large grant.wade@wsu.edu}
        \item[] {\Large robby.boney@wsu.edu}
        \item[] {\Large robert.wooner@wsu.edu}
    \end{itemize}
\end{multicols}

\vfill

\begin{flushright}
    \vspace{20mm}
    {\Large \textbf{Date:} Sunday, October 15th, 2017}
\end{flushright}

% Begining of second page
\clearpage


\tableofcontents{}

\chapter{Introduction}
The goal of this project is to create a cryptographically
secure cross platform password manager. The cross platform
compatability will be achived using electron and node.js.
This section will describe who the intended audience for the
password manager will be and describe the purpose of the
project in depth.
\section{Document Purpose}
The product we are writing this SRS document for is the cryptographically
secure cross platform password manager Version 0.0.1. This password manager 
will create strong passwords and encrypt them. It will remember the password 
for the website it is being created for.
\section{Project Scope}
This software is a password manager that creates cryptographically secure 
passwords. It will store the hashed passwords in a json file for safe keeping.
There can be multiple profiles, each one will have a master password of its own 
that will unlock the vault gaining access to the passwords. The master password
will be created by the user so that they can remember it. The password manager 
can however create a good master password that the user can write down to remember.
the master password will not be sent through email or text so that there will
be no chance of it being stolen from a malicious attacker.

Using this password manager allows the user to have strong and secure passwords 
that they will not have to remember. Not needing to remember allows for a strong 
password that has a very little chance of being cracked. The user only needs to sign
in with their master password and copy and paste the desired password from the 
vault into the website, or other password field.
\section{Intended Audience and Document Overview}
\section{Definitions, Acronyms and Abbreviations}
\section{Document Conventions}
\section{References and Acknowlegments}

\chapter{Overall Description}
\section{Product Perspective}
\section{Product Functionality}
\section{Users and Characteristics}
\section{Operating Environment}
The environment in which this software will be operating in are all
major operating systems, OS, Windows, Linux. 
\section{Design and Implementation Constraints}
The biggest constraints for this software are time and security considerations.
We need to make sure that our software follows popular, effective, and
accepted security protocols. Given the time frame in which to create this
product and the importance that our password manager successfully encrypts
and protects our passwords, we need to work hard and efficient.
\section{User Documentation}
\section{Assumptions and Dependencies}

\chapter{Specific Requirements}
\section{External Interface Requirements}
\subsection{User Interfaces}
Our product will have a login window that will have a username text box and password
text box and an enter button. The enter button will be pressed after the password 
and username are inputted. The login window will blur out the rest of the window
behind it.
We will have a minumize, maximize  and exit button in the top left corner. 
we will have tabs along the top that will take you to different password profiles,
like work, play, etc. 
There will be a create password button that will create a pop up that has entries
for a url and a drop down for picking the password profile. 
\subsection{Hardware Interfaces}
\subsection{Software Interfaces}
\subsection{Communications Interfaces}
\section{Functional Requirements}
\section{Behaviour Requirements}
\chapter{Other Non-Functional Requirements}
\section{Performance Requirements}
\section{Safety and Security Requirements}
\section{Software Quality Attributes}

\chapter{Other Requirements}

\begin{appendices}
    \chapter{Data Dictionary}
    \chapter{Group Log}
\end{appendices}



\end{document}
